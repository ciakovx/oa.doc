\documentclass{article}
\usepackage{apacite}

\usepackage{listings} % For R code
\usepackage{color} % Layout of R code

\usepackage{pgfplotstable} % Puts CSV tables in document
\usepackage{booktabs} % Nicer layout for tables


\definecolor{dkgreen}{rgb}{0,0.6,0} %colors for R
\definecolor{gray}{rgb}{0.5,0.5,0.5}
\definecolor{mauve}{rgb}{0.58,0,0.82}

\lstset{frame=tb,
  language=R,
  aboveskip=3mm,
  belowskip=3mm,
  showstringspaces=false,
  columns=flexible,
  basicstyle={\small\ttfamily},
  numbers=none,
  numberstyle=\tiny\color{gray},
  keywordstyle=\color{blue},
  commentstyle=\color{dkgreen},
  stringstyle=\color{mauve},
  breaklines=true,
  breakatwhitespace=true
  tabsize=3
}

\begin{document}


\bibliographystyle{apacite}

\title{\bf Open Access Journal Publishing at UT Arlington: \\ \large An Analysis Using Academic Analytics Data in Combination with DOAJ Data}
\author{Clarke Iakovakis}
\maketitle


\section{Objective}
To determine the scale of publishing in open access journals by UT Arlington academic departments, and the open access journals in which they publish.

\section{Introduction}
The scale of open access publishing in both green and gold forms has been steadily increasing. As more open access journals become sustainable and reputable, the scale in which researchers want to publish there increases. 
Furthermore, institutional open access policies, such as those passed by faculties at Harvard, MIT, Oregon State University, and most recently, the University of California System, will set a standard for university-supported publishing that many smaller institutions will want to copy.

Testimonies by those individuals at institutions which passed OA policies indicate the importance of building momentum and education behind the movement, in order to ensure that faculty voices are leading the call for a change in publication norms, which is critical for success. 
Therefore in the early stages of building the movement, it will be useful to determine which departments are already publishing in open access journals. 
These individuals will not only be familiar with the process of publication--which may vary somewhat from publication in toll-access journals--and perhaps more importantly may understand the virtues of open access publishing, from the increase in citation metrics to the larger practical and ethical benefits of making their research more accessible.


\section{Academic Analytics Data}
In order to discover which open access journals faculty are publishing in, one of course needs data on faculty publications. 
If your institution happens to subscribe to Academic Analytics (AA), this is one source for this data. 
Academic Analytics is a subscription database providing metrics on publication counts, citation counts, research funding, and awards to faculty. 
According to their homepage, "The Academic Analytics Database (AAD) includes information on over 270,000 faculty members associated with more than 9,000 Ph.D. programs and 10,000 departments at more than 385 universities in the United States and abroad" \cite{AcadAn}.

Academic Analytics does not provide any information as to whether or not journals are open access. 
AA does collect data on specific journals that individual faculty members publish in, as inferenced by their provision of "a \textit{numeric} tally of each faculty member’s total scholarly productivity in each of the five areas of scholarly research (journal articles, citations, books, research grants and honorific awards)" (emphasis mine); nonetheless, the micro-level data is not accessible. 
However, AA provides publication data aggregated by academic department, through the "Department Articles Market Share" page. 
For the purposes of the following study, I used only the following two variables.                             
The variables are not explicitly defined within the table, nor could I find a codebook specifically defining these particular variables, therefore the following definitions are mine:
\begin{itemize}
\item Journal Name: This is the primary key for the table. It is the name of the journal in which researchers for that department have published.
\item Unit Articles: Number of articles published in the specified journal by UT Arlington researchers, aggregated by unit (i.e. department)
\end{itemize}

 In the case of UT Arlington, this provides data on 48 departments (called "Units" in the table), including variables on the following list. 


\subsection{Scope of Academic Analytics Journal List}
First, we must establish the coverage of Academic Analytics journals. 
As Scopus is a well-regarded indexing service, the extent to which the set of AA journals is accounted for in Scopus will be telling.
Scopus provides an updated title list (http://www.elsevier.com/online-tools/scopus/content-overview), which as of February 2014 included 34,276 titles. 
The vast majority of these titles are journals; the distribution of types of titles is below:

\begin{center}
\pgfplotstabletypeset[
	col sep=comma,
	columns/Type/.style={string type},
	columns/Count/.style={string type},
	columns/PercentTotal/.style={string type},
	every head row/.style={
		before row=\toprule,
		after row=\midrule
	},
	every last row/.style={after row=\bottomrule}
]{scopus.type.tot.csv}
\end{center}

%insert R code%


Academic Analytics provides a full list of their journals through their database. 
The following data is proprietary and for internal University of Texas at Arlington staff use only.


After reading in the CSV file, it requires a bit of cleanup. First we create a vector of the title names from the Academic Analytics file, and coerce the vector to a factor. We convert it to all upper case, as some capital letter naming conventions vary 

\begin{lstlisting}
aa.journals <- read.csv(file=file.path(getwd(), "Copy of Journals_AAD2011.csv")) # read in the Academic Analytics file
aa.titles <- data.frame(aa.journals$AAD.2011.Journal.List) # get list of AA titles
aa.titles <- factor(aa.titles$aa.journals.AAD.2011.Journal.List) # convert to factor
aa.titles <- toupper(aa.titles) # convert to upper case
dupe.b <- duplicated(aa.titles) # logical vector of duplicates
aa.list <- aa.titles[!dupe.b] # return all AA journals as characters, in caps, without duplicates
aa.list.dupes <- aa.titles[dupe.b] # return all duplicated journals from the AA list (203,883)
\end{lstlisting}

I first 



it's vital to establish the extent to which the Academic Analytics journal list includes journals indexed by the Directory of Open Access Journals. To do this I downloaded the full list of journal names indexed by AA





The “Department Articles Market Share” page provides data on 48 departments (called “Units” in the table) at UT Arlington, including the variables on 





\section{Literature Review}
The UT Arlington Libraries is a strong advocate for open access to scholarly information; that is, "digital, online, free of charge, and free of most copyright and licensing restrictions."  \cite{RefWorks:102}

% \cite{RefWorks:102}









\bibliography{filename}


\appendix
\section{R Code for Open Access Journal Coverage in Academic Analytics}














\end{document}